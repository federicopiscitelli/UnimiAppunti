\begin{center}
    \textbf{--------- Lezione 5 - 9 ottobre 2020 ---------}
\end{center}

\section{Gestione dei dati di posizione}
Quando il device mobile ha calcolato la posizione, la può usare localmente oppure la può inviare ad un server.

In entrambi i casi si possono adottare tecniche per gestire il dato di posizione. 
Le più comuni sono:
\begin{itemize}
    \item Geocoding e reverse-geocoding: entrambi vengono svolti dal server, geocoding converte l'indirizzo geografico in coordinate spaziali, reverse geocoding è l'operazione inversa
    \item Geofencing: viene svolta in locale dal device, si definisce un'area e il device tiene monitorata la posizione dell'utente quando si entra e si esce dall'area (ad esempio attraverso una notifica)
    \item calcolo delle distanze: esiste una formula che calcola la distanza dalla terra (calcolo in linea d'aria)
    \item calcolo delle distanze su rete stradale: non abbiamo un calcolo in linea d'aria ma si modella la rete stradale come un grafo dove:
    \begin{itemize}
        \item i nodi sono le intersezioni
        \item gli archi sono i segmenti di strada
    \end{itemize}
    Ogni arco è etichettato con la distanza geografica tra i due nodi oppure con il tempo necessario per andare da un nodo all'altro calcolato come:
    \begin{itemize}
        \item tempo: distanza / limite di velocità (non molto affidabile, dipende dal traffico e altri fattori) 
        \item tempo di percorrenza medio di altri utenti
    \end{itemize}
\end{itemize}

Spesso i dati vengono memorizzati lato server dove è necessario definire delle tecniche di trattamento apposite.
\\ Ad esempio molti DBMS utilizzano dati spazio temporali ed offrono delle estensioni a DB classici per gestirli. 

Ci sono tre query comunemente adottate nei servizi e nel DBMS con 
estensioni spaziali:
\begin{itemize}
    \item Range query: si dà in input una zona geografica (che può essere un rettangolo oppure un cerchio) e la query ritorna tutti i punti che appartengono a quella zona indicata. Es: dammi tutte le automobili che sono nel posteggio
    \item Nearest Neighbors (NN): ho un insieme di punti e do in input uno di questi punti. La NN mi ritorna l'oggetto più vicino ad un punto o ad un altro oggetto. Es: dammi le tre stazioni di servizio più vicine a me
    \item Reverse-NN: dato un insieme di punti P e un punto q di P,  ritorna tutti i punti di P che hanno q come punto più vicino. Es: dammi tutti gli utenti che hanno me come loro utente più vicino o dammi tutte le case che hanno quel supermercato come il più vicino
\end{itemize}

I DB implementano delle strutture dati chiamati indici che permettono di rendere le operazioni più veloci. 
\\ Per esempio una range query su un’area “piccola”, rispetto alla distribuzione dei punti, ha una complessità logaritmica nel numero dei punti.




