Il corso introduce la distribuzione dei sistemi e la loro estensione ai sistemi pervasivi ottenuti dall'inclusione di oggetti smart, IOT, sensori. 
Con un sistema l'obiettivo è anche quello di non far vedere all'utente tutta la complessità del sistema, far riuscire a far sincronizzare tutti i componenti di un sistema, gestire i guasti.
\\ Creazione di un progetto di un sistema pervasivo distribuito. 

Fondamenti del corso:
\begin{itemize}
    \item organizzazione dei nodi nei sistemi e architettura client-server
    \item comunicazione, modelli di comunicazione nei sistemi distribuiti
    \item algoritmi sulla sincronizzazione tra i vari sistemi e sul tempo (mutua esclusione e algoritmi di elezione di un nodo: abbiamo un insieme di nodi, ma chi è il capo?)
    \item seminario sull'applicazione dei sistemi in Google
\end{itemize}

Modalità d'esame: 
\begin{itemize}
    \item scritto di fine appello (fine maggio/inizio giugno): domande a scelta multipla e un paio di domande aperte tra cui un esercizio
    \item progetto individuale
\end{itemize}
L'esame deve essere concluso prima del nuovo anno accademico. Il progetto viene presentato a fine aprile.
Il voto finale è calcolato come la media dei due voti.

