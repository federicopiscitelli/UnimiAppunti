\begin{comment}
    
    Omar
    Data privacy
    Importante nell'ambito dei sdp perché trattano dati che potrebbero essere sensibili.
    Cos'è la privacy? Il diritto di essere lasciato in pace, in sintesi.
    Per passare da questo alla data privacy, occorre capire cosa sono i dati personali.
    GDPR dal 2018, nell'articolo 4 si definisce dati personali come: dati/informazioni che riguardano un individuo identificato o identificabile (più debole). La capacitià di un individuo di controllare il rilascio e la distribuzione dei propri dati.
    Quasi-identifier: informazione che combinata con altra informazione può restringere i candidati ai quali i dati corrispondono.

    Servizi online: gruppo di utenti che mandano richieste. L'avversario (potrebbe essere il fornitore del servizio) guarda i dati nelle richieste (magari anche anonime, con username) e, avendo accesso all'external knowledge, si ha violazione di privacy quando da una parte si hanno i dati personali e dall'altra i dati dell'utente, associandoli.
    
    Geo-sn
    Utente, anche non essendo taggato diretatmente ad una risorsa, può essere associato attraverso la location.
    SLIDE
    
    Nei pervasivi:
    Più dati, più esposizione.
    Nuovi tipi di dati, quindi nuovi modi per re-identificare le persone e derivare dati sensibili
    Rispetto ai dati tabellari, abbiamo stream continui di dati e le sequenze possono essere correlate
    Meno consapevolezza delle persone, più difficile esprimere consenso e monitorare
    Mancanza di interfacce per controllare le preferenze
    
    Come possiamo proteggerci:
    GDPR:
    - garantire security
    - privacy by design: non progettare prima e poi pensare alla privacy, pensare alla privacy mentre si progetta
    - pseudonimitazion
    - data minimization
    
    Transaprency: non nascondere all'utente come vengono trattati i loro dati. Avere log dettagliati, blockchain è un modo
    Unlinkability: separare informazioni e non renderle riassociabili, tecniche su slide.
    Intervenability: dev'esserci un modo di intervenire nel processo. Break Glass procedures, data rectification, right to be forgotten, manual override
        
    GDPR non si applica su informazioni anonime
    
    Questi 3 sono importanti oltre a sicurezza e altri 2
    
    Data minimization: raccogliere dati alla precisione che serve per il servizio che si fornisce e non si tengono per sempre, per quello che serve e usarli per lo scopo dichiarato.
    pseudonimitazion: modo di fare unlinking, salvando un id e avendo da qualche parte il mapping. Non è anonimizzazione.
    
    Location k-anonimity: allargo la regione da avere k - 1 utenti in più e anziché mandare la posizione esatta mando l'area
    Un server fidato sa dov'è l'utente e può filtrare la risposta per dare quella più appropriata all'utente, visto che l'area è più imprecisa
    

    
    
    
    -----------------
    Cos'è la privacy?
    Il diritto di essere lasciato in pace.
    Cos'è la data privacy? 
    Abilità di controllare il rilascio, l'uso e la distribuzione di personal data (qualsiasi informazione a cui sono associate un individuo identificato o identificabile)
    
    
    
    
    
    
\end{comment}