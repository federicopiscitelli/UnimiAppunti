Il pervasive computing, anche detto ubiquitous computing, rappresenta un'area dell'informatica che è l'intersezione di diverse altre aree fondamentali tra cui:
\begin{itemize}
    \item reti
    \item sistemi
    \item algoritmi
    \item architetture
    \item intelligenza artificiale
\end{itemize}
\phantom \\

Gli elementi che caratterizzano il pervasive computing sono smart objects che possono diventare nodi di un sistema distribuito.
\\
\\
\say{The most profound technologies are those that disappear. They weave themselves into the fabric of everyday life until they are indistinguishable from it.} - Mark Weiser.\\\\
Con la parola pervasive si intende la volontà di avere un ambiente che, tramite una modifica attuata dal cumputing, riesca a rispondere alle nostre esigenze. I nodi del sistema che permettono di fare un sensing dell'ambiente e di cambiarlo, sono dispositivi harware e software distribuiti.

\newpage
\section{Sistema distribuito pervasivo}
Un sistema distribuito pervasivo è un sistema distribuito con altre caratterisiche:

\begin{itemize}
    \item i nodi del sistema sono nodi non convenzionali ovvero sono dei dispositivi con altre funzionalità principali (es. elettrodomestici con capacità di calcolo e comunicazione).
    \item il sistema è adattivo:  la logica del sistema considera il contesto attraverso un ascolto dei parametri dell'ambiente e adatta il sistema per fornire un servizio migliore.
\end{itemize}

\phantom \\

Questi sistemi sono dotati di alta volatilità poiché:
\begin{itemize}
    \item Potrebbero essere soggetti a fallimenti di dispositivi e di comunicazione molto più che nei tradizionali sistemi distribuiti poiché i nodi potrebbero essere veramente tanti (sensori) e i sistemi di comunicazione diversi.
    \item In alcune situazioni la banda per la comunicazione può essere limitata
    \item Aumentando il numero dei dispositivi, aumenta la probabilità che questi si associno o si dissocino dal sistema distribuito (sistema più dinamico che richiede meccanismi di associazione/disassociazione particolari).
\end{itemize}

\section{Mobile computing}
Il pervasive include anche la mobilità attraverso dispositivi integrati in spazi e ambienti differenti (ad esempio l'abitazione, l'ufficio, la città) e spesso questi dispositivi risultano invisibili (non hanno un'interfaccia diretta). Occorre dunque gestire gli smart spaces, quindi stabilire quale tipo di reti instaurare tra questi oggetti, come farli comunicare e in che modo gestirli. Questi sistemi crescono molto velocemente ed è fondamentale, al fine di gestirne la scalabilità, cercare di formare una rete tale per cui nodi vicini comunichino in maniera diretta al fine di ottimizzare lo scambio di messaggi all'interno dell'intera rete.
\\
\\
\textit{Uneven conditioning}: occorre convivere in un ambiente in cui cambia la connettività, in cui alcune cose sono più smart di altre. Occorre progettare un sistema in grado di funzionare anche quando è sconnesso o quando non vi sono condizioni ottimali, che si adatti anche in reti di sensori più primitivi.
\newpage
Anche nel mobile computing vi sono risorse limitate e problematiche da gestire:
\begin{itemize}
    \item tipi di interfacce diverse
    \item varianza nella connettività (ip che cambia)
    \item possibilità di lavorare in modo sconnesso (con memoria locale e meccanismi di sincronizzazione non appena la connessione è disponibile) e mobile data management
\end{itemize}
\phantom \\

Principali argomenti di ricerca:
\begin{itemize}
    \item Rete
    \item Mobile information access
    \item Mobile data management (privacy e sicurezza, mobile cloud services, LBS)
    \item Positioning (indoor, outdoor, prossimità, tracking)
    \item Software (App e mobile services design, develompent and testing, scalabilità)
\end{itemize}

\section{Pervasive Computing}
Il pervasive computing spazia tra nodi mobili, computer embedded, attuatori e sensori connessi ad internet che ascoltano ed influenzano il mondo fisico.\\
\\
Esempi di pervasive systems:
\begin{itemize}
    \item smart environment systems
    \begin{itemize}
        \item Smart home services
        \item smart energy management
        \item smart transportation (usando crowdsensing attraverso smartphone e sensori)
    \end{itemize}
    \item e-Health systems per 
    \begin{itemize}
        \item tele-healthcare
        \item independent living and ageing well
        \item accessibility
    \end{itemize}
\end{itemize}


\section{IoT}
Nell'IoT (Internet of things) sono inclusi quei dispositivi connessi ad internet che sono pensati per fare sensing o interventi. Nell'IoT ci sono moltissimi dispositivi che non hanno un indirizzo IP, partecipano a delle sotto-reti e utilizzano Internet attraverso un gateway. Nonostante il pervasive computing comprenda anche l'internet of things, non tutto ciò che è IoT è da considerare pervasive computing. E' bene sapere che, quando introduciamo nodi in un sistema distribuito, necessitiamo di un coordinamento per fare comunicare i vari dispositivi distribuiti nel modo corretto. Nella maggior parte dei casi questo coordinamento viene gestito in maniera centralizzata.

\subsection{Smart}
Molti produttori di tecnologia spacciano loro prodotti di fascia medio-alta come smart. Ma cosa è veramente "smart"?\\
Di fatto un dispositivo per essere considerato smart non basta che sia connesso ad internet e che sia comandabile da remoto. Un dispositivo smart deve infatti rispettare delle caratteristiche per essere definito tale:
\begin{itemize}
    \item Connesso ad internet: ovviamente la prima caratteristica è quella più comune, ovvero necessita di una scheda di rete che gli consenta di connettersi in rete così da poter essere acceduto dall'esterno, invocare servizi, etc..
    \item Capacità di calcolo: il dispositivo smart deve essere in grado di svolgere pre-computazione di dati o utilizzare protocolli per inviare e ricevere dati dal cloud
    \item Servizi context-aware personalizzati: la vera caratteristica principale di un dispositivo smart è quella di riuscire a capire il contesto in cui si trova e offrire servizi personalizzati in base alla situazione. Ad esempio una macchinetta del caffè che capisce dal tono della tua voce che tipo di caffè vuoi (macchiato, espresso..)
\end{itemize}

\subsubsection{"Smart" appliances}
Il campo smart si sta sempre più diffondendo in ogni settore, cercando di rendere "smart" sempre più oggetti e contesti. Ad esempio:\\ 

Smart vehicles, le quali integrano diverse compinenti:
\begin{itemize}
    \item centinaia di sensori diversi
    \item telecamere
    \item 2+ leader che ricostruiscono in 3d l'ambiente
    \item 4+ radar
    \item gps
    \item attuatori che in base ai dati rilevati e ad un algoritmo decisionale, modificano il comportamento dell'auto in base ai dati rilevati. In questo campo è sempre più necessario l'introduzione del 5g, così da riuscire in tempo reale a far comunicare l'auto con gli altri veicoli e con l'ambiente circostante. Tutti questi dati contribuiscono ad un sistema distribuito.
\end{itemize}
\phantom \\

Altro esempio è quello del sensor network in cui rientrano, ad esempio, tute piene di sensori per la lettura dei parametri vitali. 

\section{Applicazione ed ambiti del pervasive computing}
Abbiamo visto dunque come il pervasive computing sia un incrocio di diverse aree dell'informatica. Questo ambito è tutt'ora in via di sviluppo e ancora da esplorare a fondo. Ad esempio, alcune nuove questioni ancora da esplorare sono:

\begin{itemize}
    \item come usare smart spaces in modo utile ed efficiente e dunque adattare il sistema in base al contesto
    \item cercare di rendere i dispositivi sempre più invisibili e integrati all'interno dell'ecosistema
    \item dotare ogni nodo di un indirizzo e capacità specifiche e rendere l'integrazione e la comunicazione tra i vari dispositivi il più trasparente possibile. Ad esempio, entrando in una camera di hotel, il proprio telefono che conosce le tue abitudini si associa ai vari componenti smart della stanza e li dirige per far si che la tua esperienza sia il più personalizzata possibile
\end{itemize}
\phantom \\

Ecco alcuni ambiti del pervasive computing:
\begin{itemize}
    \item modellare le attività umane di alto livello
    \item crowd sensing
    \item energia
    \item pervasive health, tramite dispositivi indosabili o nell'ambiente e l'interazione dei due
    \item trasporti e ottimizzazione della mobilità
    \item edge computing, cercare di spostare il calcolo sui dispositivi vicini, soprattutto quando è oneroso
    \item security e privacy, sono molto facili da attaccare dal punto di vista del software e dell'harware su cui girano. Privacy: le attività svolte ad esempio in casa sono private e se sono condivise in cloud, qualunque attacco al cloud va ad impattare sui dati.
\end{itemize}







\begin{comment}
kikko
\section{Introduzione al pervasive computing}
Il pervasive computing (anche detto ubiquitous computing)
raprresenta una visione (area di intersezione tra aree fondamentali dell'informatica come reti, algoritmi, architetture e AI). Gli elementi del pervasive sono smart objects che rappresentano i nodi di un sd. 
\textit{Le più profonde tecnologie sono quelle che spariscono. Si nascondono negli elementi di ogni giorno fin tanto che non si riesce a distinguerle - Mark Weiser}
Nel pervasive computing ci serve avere un ambiente che risponda alle nostre esigenze: capire e modificare l'ambiente (capting and computing).

Un SDP è un SD con altre caratteristiche
\begin{itemize}
    \item nodi non convenzionali (non sono unità di calcolo, non sono computer), nei quali sono stati aggiunte capacità di calcolo e comunicazione.
    \item adattività: la logica del sistema considera il contesto e adatta il comportamento del sistema per ottimizzare l'obiettivo finale
\end{itemize}

Questi sistemi sono dotati di alta volatibilità poichè:
\begin{itemize}
    \item ancora di più che negli SD tradizionali potrebbero essere soggetti a fallimenti di dispositivi e di comunicazione
    \item può cambiare la caratteristica della comunicazione (la larghezza di banda può essere limitata ad esempio)
    \item aumentando di numero i dispositivi possono avere difficoltà a creare o distruggere associazioni tra le componenti sw dei dispositivi
\end{itemize}

SCHEMA DA SD a SDP

\subsection{Mobile Computing}
Principali problemi:
\begin{itemize}
    \item risorse limitate
    \item tipi di interfacce diverse
    \item varianza nella connettività
    \item locazione variabile
\end{itemize}

Principali argomenti di ricerca:
\begin{itemize}
    \item Rete
    \item Mobile information access
    \item Mobile data management (privacy e sicurezza, mobile cloud services, LBS)
    \item Positioning (indoor, outdoor, prossimità, tracking)
    \item Software (App e mobile services design, develompent and testing, scalabilità)
\end{itemize}

\subsection{Pervasive Computing}
Da nodi mobili a computer embedded e attuatori e sensori connessi ad Internet, che ascoltano e influenzano il mondo fisico. 

Esempi di pervasive systems:
\begin{itemize}
    \item smart environment systems
    \begin{itemize}
        \item Smart home services
        \item smart energy management
        \item smart transportation (usando crowdsensing attraverso smartphone e sensori)
    \end{itemize}
    \item e-Health systems per 
    \begin{itemize}
        \item tele-healthcare
        \item independent living and ageing well
        \item accessibility
    \end{itemize}
\end{itemize}

\subsection{IoT}
Dispositivi che sono pensati per fare sensing o interventi, connessi a internet.
Si dice IoT anche quando il dispositivo non ha un IP Address.

Non tutto ciò che è IoT è pervasive.

Molti produttori di tecnologia spacciano loro prodotti di fascia medio-alta come smart. Ma cosa è veramente "smart"?
\begin{itemize}
    \item Connessa a internet (accesso remoto, invocazione di servizi, cooperazione ...)
    \item Esegue algoritmi (localmente o remotamente) per analizzare dati e comprendere il contesto attraverso l'integrazione con AI
    \item Offre servizi context-aware personalizzati (deve capire il contesto e offrire servizi personalizzati in base a quello)
\end{itemize}

\subsection{Reti di sensori}
Sono importante componenti dei SDP. I sensori possono essere parte di oggetti smart (smartphone, macchina).
Domande riguardanti le reti di sensori:
- 
-
-

Nuove questioni da esplorare nell'ambito del pervasive computing:
\begin{itemize}
    \item smart spaces devono essere usati efficacemente (adattività, context-awareness, anticipazione dei bisogni)
    \item invisibilità: l'interazione con gli utenti deve essere minimizzata
    \item le risorse in un ambiente pervasivo dovrebbe essere "scopribile" e bisogna abilitare associazioni e collaborazioni dinamiche.
\end{itemize}
Contesti di ricerca nel pervasive computing:
\begin{itemize}
    \item Modellazione e Riconoscimento del contesto e delle attività
    \item Crowd Sensing
    \item Energy Analytics
    \item Pervasive Health
    \item Pervasive Transportation
    \item Edge Computing
    \item Security and Privacy
\end{itemize}


___----____----____----___----____-_____--___

Edoardo Rigolini

Il pervasive computing, anche chiamato ubiquitous computing, rappresenta un'area dell'informatica che è l'intersezione di diverse altre aree fondamentali (reti, sistemi, algoritmi, architetture e intelligenza artificiale). Elementi del pervasive sono smart objects che possono diventare nodi di un sistema distribuito. 
Un grande risultato sarebbe far scomparire il computing, in modo che sia integrato all'interno dell'ambiente - osservazione di Mark Weiser.
Nel pervasive, vogliamo avere un ambiente che risponda alle nostre esigenze. Si vuole quindi capire e modificare l'ambiente attraverso il computing. Questi nodi del sistema che permettono di fare un sensing dell'ambiente e di cambiarlo, sono dispositivi hw e sw distribuiti. 
Distributive pervesavi system: sistema distribuito con nodi non convenzionali (dispositivi con altre funzionalità principali, ad es. elettrodomestici, con capacità di calcolo e comunicazione).
Questo non è sufficiente, oltre ad avere caratteristica di sistema distribuito (fornire una funzionalità senza far vedere i componenti che partecipano a realizzarla), c'è anche adattività: dobbiamo capire il contesto corrente attraverso un ascolto dei parametri dell'ambiente e adattare il sistema per fornire un servizio migliore.
Questi sistemi sono, più dei sistemi distribuiti tradizionali, soggetti a failure, poiché i nodi potrebbero essere veramente tanti (sensori) e i sistemi di comunicazione sono diversi. Non tutti i dispositivi hanno un proprio IP.
In alcune situazioni la banda per la comunicazione può essere limitata.
Aumentando il numero dei dispotivi, aumenta la probabilità che questi si associno o si dissocino dal sistema distribuito (sistema più dinamico che richiede meccanismi di associazione/disassociazione particolari).

Il pervasive include anche la mobilità, ma questi dispositivi sono integrati in spazi (ad esempio l'abitazione, l'ufficio, la città) e spesso sono invisibili (non hanno un'interfaccia diretta). Occorre gestire gli smart spaces, quindi stabilire quale tipo di reti instaurare tra questi oggetti, ecc. Questi sistemi crescono molto velocemente ed è fondamentale, al fine di gestire la scalabilità, considerare quelli che sono vicini tra di loro e ottimizzare la comunicazione tra questi. (?)
Uneven conditioning: occorre convivere in un ambiente in cui cambia la connettività, in cui alcune cose sono più smart di altre. Occorre progettare un sistema in grado di funzionare anche quando è sconnesso o quando non vi sono condizioni ottimali, che si adatti anche in reti di sensori più primitivi (?).

Anche nel mobile computing vi sono risorse limitate e problematiche da gestire: diversi tipi di interfaccia, ip che cambia, possibilità di lavorare in modo sconesso (con memoria locale e meccanismi di sincronizzazione non appena la connessione è disponibile) e mobile data management. Nel mobile computing vi sono anche tecniche di positioning.


Nell'IoT ci sono moltissimi dispositivi che non hanno un indirizzo IP. Partecipano a delle sotto-reti e utilizzano Internet attraverso un gateway. 

Non tutto ciò che è IoT è pervasive computing. Quando introduciamo nodi in un sistema distribuito, necessitiamo di un coordinamento. Nella maggior parte dei casi c'è una gestione centralizzata. (?)

Smart:
- Connesso ad internet, consentendo i accedere dall'esterno, invocare servizi, etc.
- Deve avere capacità di calcolo. Pre-computazione di dati o mandarli in cloud e ricevere dati
- Offre servizi personalizzati e basati sul contesto




Smart vehicles:
- centinaia di sensori diversi
- telecamere
- 2 leader che ricostruiscono in 3d l'ambiente
- 4 radar
- gps
- ...
- attuatori, che in base ai dati rilevati e all'algoritmo che decide, avendo processati questi dati, cosa fare, modificando il comportamento dell'auto.
Con il 5g e reti ad hoc possono comunicare con altri veicoli e dispositivi lungo la strada. Tutti questi dati contribuiscono ad un sistema distribuito.




Sensor network:
Esempio la tuta nella slide precedente: bus e synch che riceve i dati e li comunica in rete. I sensori di per sé non sono in grado di comunicare, necessitano ... (?)

Pervasive computing: incrocio di varie aree dell'informatica. Nuove questioni da esplorare:
- come usare smart spaces in modo utile ed efficiente. Capire il contesto e in base a questo adattare il sistema.
- invisibilità di questi oggetti, l'interazione dev'essere minimizzata
- ciascun nodo avrà un indirizzo, delle capacità, etc. Quando si entra in un nuovo ambiente, devono essere scoperti i dispositivi presenti e devono essere associati ai propri.

Ambiti del pervasive computing
- modellare le attività umane di alto livello
- crowd sensing
- energia
- pervasive health, tramite dispositivi indosabili o nell'ambiente e l'interazione dei due
- transportation, ottimizzare la mobilità
- edge computing, cercare di fare calcolo, soprattutto quando molto oneroso, ... (?)
- security e privacy, sono molto facili da attaccare dal punto di vista del sw e hw su cui girano. Privacy: le attività svolte ad esempio in casa sono private e se sono condivise in cloud, qualunque attacco al cloud va ad impattare sui dati.



\end{comment}