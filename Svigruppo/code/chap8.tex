\section{Docker}

Con la crescita della dimensione degli oggetti da sviluppare e del numero di persone concorrenti, è nata la necessità di introdurre nuovi strumenti per poter comunque lavorare in maniera efficace. Quando si sviluppano software complessi, possono essere impiegati diversi:

\begin{itemize}
    \item linguaggi (polyglot)
    \item librerie (versioni multiple)
    \item servizi (dbms, httpd, cache, log, ...)
    \item persone (onboarding)
    \item testing (fixtures)
\end{itemize}

\noindent Per risolvere questo problema si è pensato di adottare un approccio DevOps, scaricando alcune responsabilità sul developer. Si vuole:

\begin{itemize}
    \item Dal punto di vista dei Dev:
    \begin{itemize}
        \item predicibilità: eseguendo due volte lo stesso codice, sullo stesso ambiente di sviluppo, si ottiene lo stesso risultato.
        \item replicabilità: spostandolo su un altro ambiente viene mantenuto il funzionamento.
        \item versionamento: sistema di gestione delle configurazioni (il software funziona con un certo linguaggio, certe librerie, ecc.)
        \item leggero: tutto questo dev'essere ottenuto attraverso un meccanismo efficiente e leggero.
    \end{itemize}
    \item Dal punto di visto degli Ops:
    \begin{itemize}
        \item tolleranza al fallimento: replicare le parti in maniera tale che se una parte smette di funzionare, le altre continuano ad andare.
        \item portabilità: funzioni anche sul proprio server, oltre a quello di sviluppo.
        \item scalabilità: possibilità di aumentare le risorse con l'aumentare dei clienti. Più risorse, più performance.
        \item componibilità, attraverso moduli.
        \item schedulazione \& orchestrazione, in maniera automatica.
    \end{itemize}
\end{itemize}

\noindent Possibili soluzioni sono:

\begin{itemize}
    \item Configuration management: strumenti che permettono di dare una descrizione, in alcune casi imperativa, per configurare macchine hardware. Ansible, Chef, Puppet, Saltstack, Terraform.
    \item Virtualizzazione: strumenti che virtualizzano l'hardware, permettendo di avere più sistemi operativi sulla stessa macchina. LXC, KVM, Vagrant.
    \item Containers: strumenti basati sull'idea di conteinerizzazione. Docker, Singularity.
\end{itemize}

\subsection{Virtualizzazione in stile Docker}

In Docker si ha una virtualizzazione fortemente dipendente dal sistema operativo. Necessita, quindi, di supporto da parte del kernel per:
\begin{itemize}
    \item separazione dei processi: permette la virtualizzazione dello spazio dei processi. Mantiene separati processi di gruppi diversi.
    \item supporto sul filesystem: una sola copia del filesystem che serve a più immagini diverse. Il kernel controlla che vengano allocate solo le modifiche da parte dei diversi container rispetto all'immagine standard (copy-on-write).
    \item networking: meccanismi di astrazione del networking. Container sulla stessa macchina o su macchine distinte possono lavorare insieme.
\end{itemize}

%Docker su apple o windows c'è un hypervisor linux con docker (?). Docker è efficiente se utilizzato su Linux.

\subsection{Immagini e Container}

Un'immagine, nel gergo di Docker, è un grafo aciclico (DAG) di filesystem read-only e copy-on-write. Queste possono essere ottenute attraverso: pull (da un Docker Hub, su cui è anche possibile effettuare una push), build o commit.

\begin{minted}[bgcolor=LightGray, fontsize=\footnotesize]{bash}
$ docker pull hello-world
\end{minted}

\noindent Un container è ciò che si ottiene prendendo un'immagine read-only, tramutando il filesystem in read-write e mettendo in esecuzione un processo distinto. Un container si può ottenere:
\begin{itemize}
    \item esplicitamente attraverso una create
\begin{minted}[bgcolor=LightGray, fontsize=\footnotesize]{bash}
$ docker create --name dt-image hello-world
\end{minted}
    \item implicitamente attraverso una run (= create + start)
\begin{minted}[bgcolor=LightGray, fontsize=\footnotesize]{bash}
$ docker run hello-world
\end{minted}
\end{itemize}

\noindent Comandi per la gestione delle immagini: 
\begin{minted}[bgcolor=LightGray, fontsize=\footnotesize]{bash}
docker images (elenco immagini), docker inspect, docker rmi (rimuovere immagine),
docker pull, docker push, docker export, docker import, docker create, docker run
\end{minted}

\noindent Comandi per la gestione dei container:
\begin{minted}[bgcolor=LightGray, fontsize=\footnotesize]{bash}
docker ps [-a] (elenco container; l'opzione -a visualizza anche i processi terminati),
docker inspect, docker rm (rimuovere container), docker start, docker stop,
docker kill, docker pause, docker unpause
\end{minted}

\noindent In che modo è possibile produrre un'immagine? Di seguito due possibili approcci:

\begin{itemize}
    \item worst practice: eseguire un'immagine esistente, fare delle modifiche ed infine effettuare una commit (registrandola con una tag simbolica). Questo approccio non è riproducibile.
    \item best practice: utilizzare un processo di build che si basa su un Dockerfile, un file che specifica i passi di creazione e di configurazione dell'immagine.
\end{itemize}

\noindent È poi ragionevole rendere l'immagine pubblica. Per farlo, ci si può registrare su Docker ed inviare l'immagine a Docker Hub utilizzando il comando docker push, che pusha solo i nodi che non sono già noti e tutto il necessario per poter porre in esecuzione l'applicazione in un secondo momento.\\

\noindent Un container Docker può essere posto in esecuzione in diversi modi:
\begin{itemize}
    \item ephemeral
\begin{minted}[bgcolor=LightGray, fontsize=\footnotesize]{bash}
$ docker run --rm busybox date
\end{minted}
    Eseguo un container specificando il comando da eseguire al suo interno. L'opzione --rm indica che quando il container termina viene eliminato.
    \item attached
\begin{minted}[bgcolor=LightGray, fontsize=\footnotesize]{bash}
$ docker run -it busybox
\end{minted}
    Eseguo un container allocando una TUI in cui poter eseguire i comandi.
    \item daemonized
\begin{minted}[bgcolor=LightGray, fontsize=\footnotesize]{bash}
$ docker run --name dt-daemon -d busybox \
    sh -c 'while true; do date; sleep 10; done'
\end{minted}
    Eseguo un container in background. Usualmente nei container vengono messi in esecuzione dei server, che rimangono costantemente in esecuzione, per cui questo è uno dei modi più ragionevoli. 
\end{itemize}

\noindent Ho diversi modi per poter porre in esecuzione qualcosa e ritornare in contatto una volta staccato:

\begin{itemize}
    \item ottenere i log delle istanze in esecuzione e terminate (metodo standard).
\begin{minted}[bgcolor=LightGray, fontsize=\footnotesize]{bash}
$ docker logs dt-daemon
\end{minted}
    \item attaccarsi allo stdin/out del (processo in un) container in esecuzione.
\begin{minted}[bgcolor=LightGray, fontsize=\footnotesize]{bash}
$ docker attach --sig-proxy=false dt-daemon
\end{minted}
    \item eseguire un altro processo in un container in esecuzione.
\begin{minted}[bgcolor=LightGray, fontsize=\footnotesize]{bash}
$ docker exec dt-daemon ps -a
\end{minted}
\end{itemize}

\subsection{Persistenza e comunicazione}

Il modo più ragionevole per rendere persistenti le informazioni in Docker è l'utilizzo di un volume. 
Un volume è un'astrazione di un volume fisico (un piccolo filesystem). 

\begin{minted}[bgcolor=LightGray, fontsize=\footnotesize]{bash}
$ docker volume create dt-volume
\end{minted}

\noindent Il volume è nel filesystem della macchina ospite, per cui è molto difficile condividerlo tra container su macchine diverse. Può essere letto e scritto (anche con esecuzioni ephemeral) nel seguente modo:

\begin{minted}[bgcolor=LightGray, fontsize=\footnotesize]{bash}
$ docker run --rm --volume dt-volume:/data busybox sh -c 'date > /data/file.txt'
$ docker run --rm --volume dt-volume:/data busybox sh -c 'cat /data/file.txt'
\end{minted}

\noindent L'opzione --volume monta il volume dt-volume nel mount point /data.\\

\noindent I comandi per la gestione dei volumi sono:

\begin{minted}[bgcolor=LightGray, fontsize=\footnotesize]{bash}
docker volume ls, docker volume rm, docker volume inspect
\end{minted}

\noindent Per poter operare sul filesystem host, si può, anche se sconsigliato, mappare una directory del filesystem host (quello su cui viene eseguito Docker) all'interno del container.

\begin{minted}[bgcolor=LightGray, fontsize=\footnotesize]{bash}
$ docker run -it --rm -v $(pwd):/data busybox sh -c 'date > /data/file.txt'
\end{minted}

\noindent Esiste inoltre il comando cp per copiare i file dai volumi di Docker al filesystem e viceversa. Questo ha come target un container.

\begin{minted}[bgcolor=LightGray, fontsize=\footnotesize]{bash}
$ CID=$(docker create --volume dt-volume:/data busybox)
$ docker cp $CID:/data/file.txt file2.txt
$ docker cp file2.txt $CID:/data/file3.txt
$ docker rm $CID
$ docker run --volume dt-volume:/data busybox cat /data/file3.txt
\end{minted}

\noindent Per porre in comunicazione container diversi si utilizzano le reti, con cui creiamo un segmento di rete al quale diamo un nome e in cui mettiamo dei container.

\begin{minted}[bgcolor=LightGray, fontsize=\footnotesize]{bash}
$ docker network create dt-network
$ docker run --network dt-network --name dt-nc --rm -it busybox nc -l -p 80
\end{minted}

\noindent Il nome che ha un container sulla rete è il suo stesso nome. È possibile comunicare con un container a partire da un container sulla stessa rete.

\begin{minted}[bgcolor=LightGray, fontsize=\footnotesize]{bash}
$ docker run -it --rm --network dt-network busybox sh -c 'date | nc dt-nc 80'
\end{minted}

\noindent Per rendere pubblico un servizio di rete in un container, è possibile pubblicare una porta (-p) nel seguente modo:

\begin{minted}[bgcolor=LightGray, fontsize=\footnotesize]{bash}
$ docker run --name dt-nc --rm -it -p 80 busybox nc -l -p 80
\end{minted}

\noindent I comandi per la gestione delle reti sono:

\begin{minted}[bgcolor=LightGray, fontsize=\footnotesize]{bash}
docker network ls, docker network rm, docker network inspect
\end{minted}

\subsection{Docker dal punto di vista dei developer}

Dal punto di vista del developer possiamo:

\begin{itemize}
    \item utilizzare Docker per testare la compilazione/esecuzione con development kit diversi, in maniera veloce.
    \item utilizzare l'IDE all'interno di un container, mettendolo in esercizio con tutto quel che serve e velocizzando quindi l'onboarding.
\end{itemize}