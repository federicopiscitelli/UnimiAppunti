\section{Retrospettive agili}

"Un ritrovo rituale di una comunità alla fine di un progetto per riguardare gli eventi e imparare dall'esperienza. Nessuno ha conoscenza dell'intero progetto, ogni persona ha solo un pezzo della storia. Questo rito collettivo ci aiuta a raccontare la storia e a estrarre esperienza per la saggezza."\\

\noindent Il concetto di retrospettiva è presente in tutte le metodologie agili. Questo è il mezzo principe con cui il team ragiona su cosa sta accadendo e su come potersi migliorare. Finita l'iterazione classica, prima di proseguire, occorre fare una retrospettiva (relativamente di breve durata), al fine di adattarsi a ciò che è successo nell'iterazione precedente:

\begin{itemize}
    \item capire diversi punti di vista
    \item cercare di forzare un modo di pensare
    \item favorire una discussione su cosa occorre fare e focalizzarsi su azioni specifiche (non generiche)
\end{itemize}

\noindent Le retrospettive si compngono di 5 fasi che si integrano con l'iterazione classica delle metodologie agili:

\begin{itemize}
    \item setup
    \item gather data (raccogliere dati)
    \item make insights (generare comprensioni migliori)
    \item decide what to do
    \item close
\end{itemize}

\subsection{Setup}

Si introducono al gruppo lo scopo, l'argomento e il tempo a disposizione. Mira a creare un'atmosfera dove le persone si sentono a loro agio a discutere le questioni.\\

\noindent Le attività svolte in questa fase sono:

\begin{itemize}
    \item checkin, fare una domanda semplice a cui ognuno deve rispondere con una o due parole (come ti senti rispetto a questa retrospettiva? arrabbiato, felice, ottimista, triste, preoccupato), cominciando in questo modo a raccogliere l'umore del gruppo.
    \item working agreements: mettersi d'accordo su alcune regole. Non troppe, ma ben chiare, che possono cambiare da una riunione all'altra. Si divide in gruppi e ognuno propone delle regole e dei comportamenti da seguire. Quando si discute, bisogna assumere che ognuno abbia fatto il massimo. 
\end{itemize}

\noindent Un'altra possibile attività è ESDP, in cui ciascuno si deve caratterizzare, in maniera anonima, rispetto a ciò di cui si sta parlando come:

\begin{itemize}
    \item esploratore: ha voglia di capire
    \item shopper: attende che esca qualcosa di utile
    \item in vacanza: finge di far presenza
    \item prigioniero: avrebbe preferito fare altro, ma è stato costretto a venire
\end{itemize}

\subsection{Gather data}

Non tutti hanno visto tutto. Bisogna creare un vista collettiva di cosa è successo, considerando due tipi di dati:
\begin{itemize}
    \item hard data
        \begin{itemize}
            \item eventi: riunioni, decisioni importanti, assunzione/licenziamento di membri del team, adozione di nuove tecnologie, ecc...
            \item metriche: dati come velocity, numero di segnalazioni di bug, quanto refactoring è stato fatto, schema che indica a che punto si è del progetto, ecc..
        \end{itemize}
    \item soft data: sentimenti, emozioni, umore. Quello che uno ritiene importante o che ha colpito. Non serve che tutti lo considerino importante, ma il fatto che molti non lo considerino importante potrebbe essere un dato aggiuntivo. 
\end{itemize}

\noindent Possibili attività svolte in questa fase sono:

\begin{itemize}
    \item timeline: funziona bene per rilevare hard data. Ognuno propone con delle carte eventi hard per lui significativi e poi vengono posizionate su una timeline in ordine cronologico. Dopodiché ognuno riguarda tutti gli eventi per vedere ciò che hanno scritto gli altri e cogliere ciò che è successo e di cui non ci si è resi conto. Questo può essere utilizzato anche per le emozioni, ma è meglio separare le due cose.
    \item color code dots: dopo aver raccolto hard data, vengono associati soft data mediante una votazione, su come ci si sente rispetto a quegli eventi. Ogni partecipante riceve un certo numero di gettoni colorati da posizionare sulle schede. Colori che rappresentano le emozioni o l'aspetto che si vuole mettere in evidenza.
\end{itemize}

\subsection{Make insights}

Ci si domanda "perchè?" per riuscire a trovare comprensioni maggiori di quello che è accaduto, elaborando i dati e traendone informazioni. Trovare situazioni ricorrenti, cambiamenti improvvisi di umore. È importante affrontare questa riunione con la mente molto aperta, per non lasciarsi condizionare dai propri pregiudizi.\\

\noindent Possibili attività svolte in questa fase sono:

\begin{itemize}
    \item Patterns \& Shifts: si discute lo schema ricavato precedentemente e si cercano connessioni tra eventi, situazioni ricorrenti, similitudini e cambiamenti improvvisi.
    \item Five Whys: serve a identificare diverse cause più profonde. Ci si divide a coppie, si analizza, ci si chiede reciprocamente il perché più volte (5), andando a cercare le cause prime, e dopodiché si uniscono. Il fatto di farlo a coppie e ricongiungere le questioni, fa emergere più effetti scatenanti, non fermandosi alla prima causa.
\end{itemize}

\noindent Possono essere analizzate anche le cose positive, non solo quelle negative, cercando di capire perché qualcosa ha funzionato per far in modo che la volta successiva funzioni ancora.

\subsection{Decide what to do}

Sui temi identificati precedentemente si cerca di trovare delle idee di intervento, da adottare nel prossimo progetto o nella prossima iterazione, selezionando quelle più importanti e creando dei piani di sperimentazione, preferibilmente misurabili.\\

\noindent Possibili attività svolte in questa fase sono: 

\begin{itemize}
    \item triple nickels: si sviluppano idee raccogliendo diversi contributi. In maniera circolare ognuno (3 persone) scrive la propria idea su un foglietto e lo passa alla persona successiva. Questa la guarda ed aggiunge qualche particolare, e così via. In questo modo si affronta da diversi punti di vista la stessa idea, in diverse fasi di elaborazione. Le idee devono poi essere giudicate e votate.
    \item SMART goals: le idee venute fuori per essere pianificate devono essere trasformate in SMART goal che soddisfano queste caratteristiche:
    \begin{itemize}
        \item Specific: non generico
        \item Measurable: e quindi controllabile
        \item Attainable: effettuabile
        \item Relevant: significativo
        \item Timely: opportuno... adesso
    \end{itemize}
\end{itemize}

\subsection{Close}

\begin{itemize}
    \item Occorre decidere come documentare quanto fatto (fare foto alla lavagna, raccogliere foglietti, ecc...)
    \item Fare una breve retrospettiva della retrospettiva stessa. Cosa è piaciuto? Cosa ha funzionato? Per migliorarsi.
    \item Ci deve essere un momento per fare apprezzamenti, per finire in maniera positiva.
\end{itemize}

\subsection{Logistica}

Quanto dura? \\

\noindent Non ha una durata fissa. Di solito tra le 2 e le 4 ore, ma può durare da mezz'ora a 4 giorni (magari nel caso di un progetto biennale), con le seguenti percentuali:

\begin{itemize}
    \item Set the stage: 5\%
    \item Gather data: 30-50\%
    \item Generate insights: 20-30\%
    \item Decide what to do: 15-20\%
    \item Close the retrospective: 10\%
    \item Shuffle time: 10-15\%
\end{itemize}
    
\noindent Dove si svolge? \\

\noindent Svolgerla nell'ambiente di lavoro permette di mantenere il legame con il lavoro, avendo già a disposizione i dati. Andare da un'altra parte consente di distaccarsi dall'iterazione stessa e ragionare da un punto di vista fresco, soprattutto se quell'ambiente ricorda problemi e fallimenti.

\begin{comment}
Non ha durata fissa. Di solito tra le 2 e 4 ore, comunque da mezz'ora a 4 giorni (retrospettiva magari di un progetto biennale). 
Dove si svolge? 
\end{comment}