\section{La cattedrale, il bazaar e altri modelli}
Nel libro \textit{The mythical man-month} (lettura obbligatoria fino al capitolo 7) Fred Brooks racconta la sua esperienza all'interno di IBM, durante il progetto di OS/360.\\\\
\say{Large-system programming has over the past decade
been such a tar pit, and many great and powerful beasts
have thrashed violently in it. Most have emerged with running systems—few have met goals, schedules, and budgets.
Large and small, massive or wiry, team after team has become entangled in the tar. No one thing seems to cause the
difficulty—any particular paw can be pulled away. But the
accumulation of simultaneous and interacting factors brings
slower and slower motion. Everyone seems to have been
surprised by the stickiness of the problem, and it is hard to
discern the nature of it. But we must try to understand it
if we are to solve it.} \cite{manmonth} - Fred Brooks, The mythical man-month
\subsection{I problemi identificati da Brooks}
Nonostante l'insuccesso di questo progetto, Brooks è riuscito a trarre ottime osservazioni, utili ancora oggi nel mondo dell'ingegneria del software. Le più importanti:
\begin{itemize}
    \item le tecniche di stima sono poco sviluppate e si tende ad assumere che tutto andrà bene (il modello COCOMO è la tecnica di stima più famosa in ambito software)
    \item si confonde "effort" con "progress", ma questi non corrispondono. È molto facile stimare quanto si è lavorato, è meno facile misurare di quanto si è progredito, e questo può causare ulteriori ritardi
    \item il progredire dello sviluppo viene controllato in maniera molto superficiale
    \item si risponde ai ritardi aggiungendo personale
\end{itemize}

\noindent Tra tutte queste osservazioni, l'ultima è forse la più importante, perchè è proprio da questa che deriva la famosa \textbf{legge di Brooks}. Nello sviluppo software, non tutto è facilmente parallelizzabile. Se una donna ci mette 9 mesi a far nascere un bambino, mettendone insieme due non ci vorranno 4 mesi e mezzo.

\subsection{La legge di Brooks}
\begin{center}
\say{Adding manpower to a late software project makes it later}
\end{center}
Aggiungere forza lavoro ad un progetto software in ritardo, lo fa ritardare ancora di più. Può sembrare un paradosso, eppure è così.\\\\
Ogni lavoratore deve essere formato e deve essere a conoscienza delle tecnologie utilizzate, degli obiettivi, della strategia, di quello che è stato fatto fino ad ora. Questa parte di formazione non può essere divisa e varia in base al numero di lavoratori. Inoltre tutto ciò peggiora la comunicazione tra i vari membri dei team. Se ogni parte del lavoro deve essere cordinata in maniera separata dalle altre parti l'effort aumenta a \(\frac{n(n-1)}{2}\). Tre lavoratori richiedono tre volte tanto il lavoro di comunicazione che avviene tra una coppia di lavoratori. Quattro ne richiedono sei volte tanto.
\begin{center}
    costo del coordinamento \(\propto\) \(n^2\)
\end{center}

\subsection{Modelli organizzativi}
Secondo Brooks è fondamentale preservare l'integrità concettuale di un progetto. I modelli organizzativi citati sono:
\begin{itemize}
    \item La Cattedrale: rigorosa separazione fra lavoro architetturale (accentrato) e implementativo (distribuito). Un modello gerarchico in cui si distribuisce il lavoro in maniera piramidale.
    \item La sala operatoria (H. Mills): un chirurgo e un co-pilota, contornati da una equipe con ruoli precisi, ma tutti giocano per loro.  Il vero lavoro viene svolto solamente dal chirurgo.
\end{itemize}
